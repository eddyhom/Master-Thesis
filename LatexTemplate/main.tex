% Document type
\documentclass[10pt, titlepage, a4paper, times]{article}
	% Packages
        \usepackage{graphicx}
	\usepackage{amsmath}
	\usepackage{amsfonts}
	\usepackage{fancyhdr}
	\usepackage{enumerate}
	\usepackage{listings}
	\usepackage[titletoc]{appendix}
	\usepackage[pdfborder={0 0 0},colorlinks=true, urlcolor=blue, citecolor=red, bookmarks=false]{hyperref}
	\usepackage[margin=3cm]{geometry}
\usepackage[absolute]{textpos}
	\usepackage[section]{placeins}
	\usepackage{url}
	\usepackage{tabularx}
%	\usepackage{gensymb}
	\usepackage{caption}
%	\usepackage{xltxtra}
	% Page style
	\pagestyle{fancy}
	\marginparsep = 0pt

	% Set font
	%\setromanfont{Calibri}

	\renewcommand\contentsname{Table of Contents}
	\newcommand{\HRule}{\rule{\linewidth}{0.5mm}}
	
	\newcommand{\circR}{\textsuperscript{\textregistered}}

        % set the style for the section and subsection headers
        \renewcommand{\thesection}{\thepart \arabic{section}.}
        \renewcommand{\thesubsection}{\thepart \arabic{section}.\arabic{subsection}.}

	% Code style
	\lstset{
		backgroundcolor=\color[rgb]{0.92,0.92,0.92},
		basicstyle=\footnotesize,
		showspaces=false,
		showstringspaces=false,
		showtabs=false,
		tabsize=2,
		captionpos=b,
		breaklines=false,
		keywordstyle=\color[rgb]{0,0,1},
		commentstyle=\color[rgb]{0.133,0.545,0.133}
	}

\makeatletter

\newcommand\frontmatter{%
    \cleardoublepage
  %\@mainmatterfalse
  \pagenumbering{roman}}

\newcommand\mainmatter{%
    \cleardoublepage
 % \@mainmattertrue
  \pagenumbering{arabic}}

\newcommand\backmatter{%
  \if@openright
    \cleardoublepage
  \else
    \clearpage
  \fi
 % \@mainmatterfalse
   }

\makeatother

\begin{document}
\begin{titlepage}
% Title
%\title{	

\begin{center}
		\begin{figure}[t]	
				\includegraphics[width=15mm, bb=0 0 100 100]{MDHlogga.png}
		\end{figure}
                 	\Large M\"{a}lardalen University \\
			\Large School of Innovation Design and Engineering \\
                        \Large V\"{a}ster\r{a}s, Sweden\\

                        \noindent\makebox[\linewidth]{\rule{\textwidth}{0.4pt}}\\[0.5cm]
                
                %The complete name of the course you are enrolled in
                \Large{Thesis for the Degree of Master of Science in Engineering - Robotics 30.0 credits}\\[2.0cm]

			\huge \textbf{\uppercase{Master Thesis Title}} \\ [2.5cm] %TITLE!!!!!!!!
				
			\LARGE Student Name   \\	
        	\large student\_email@student.mdh.se \\[2.5cm]			

\begin{flushleft}
			\Large Examiner: \begin{minipage}[t]{0,7\textwidth}\Large Examiner\_Name\\\large M\"{a}lardalen University, \large V\"{a}ster\r{a}s, Sweden\\ \end{minipage}\\[0.5cm]
		%	\Large M\"{a}lardalen University\\
		%	\Large V\"{a}ster\r{a}s, Sweden\\[1.0cm]
			
			\Large Supervisors: \begin{minipage}[t]{0,7\textwidth}\Large Supervisor\_Name\\\large M\"{a}lardalen University, \large V\"{a}ster\r{a}s, Sweden \end{minipage} \\[0.5cm]
		%	\Large M\"{a}lardalen University\\
		%	\Large V\"{a}ster\r{a}s, Sweden\\[0.5cm]
			
                        \Large Company supervisor: \begin{minipage}[t]{0,6\textwidth}\Large Supervisor Name, \\\large Company name, location \end{minipage}
		%	\Large Company\_Name\\
		%	\Large Location
\end{flushleft}

              \vspace*{\fill}
                    \large \today		%today should be replaced by the date the report is sent for examination

\end{center}
\end{titlepage}
%\date{}
%\maketitle

% Page style
\thispagestyle{fancy}
\fancyhead[R]{Short title of the thesis}
\fancyhead[L]{Authors}
\fancyfoot[L]{}
%\fancyfoot[LE,RO]{\thepage}
\renewcommand{\headrulewidth}{0.4pt}
\renewcommand{\footrulewidth}{0.4pt}

% Begin actual text



\def\abstract{
  \vfil
\begin{center}%
{\bfseries\abstractname\vspace{-.5em}}
\end{center}
\itshape
}

\def\endabstract{\par
}

\frontmatter


\section*{Introduktion: om begrepp som används i mallen}

I mallen anv\"{a}nder vi ett antal begrepp som det \"{a}r viktigt att ha klart f\"{o}r sig vad de avser och hur de relaterar till varandra. Vi illustrerar detta med exempel.
Du kan ha f\r{a}tt ditt examensarbete som ett uppdrag fr\r{a}n t ex ett f\"{o}retag. I s\r{a} fall har du ofta f\r{a}tt ett problem som f\"{o}retaget upplever och som du ska f\"{o}rs\"{o}ka hitta l\"{o}sning p\r{a}. Problemet utg\"{o}r i detta fall bakgrunden till syftet med arbetet och den fr\r{a}gest\"{a}llning som du arbetar fram.
Exempel: F\"{o}retaget X har ett system som de vill kunna anv\"{a}nda i en realtidstill\"{a}mpning, men prestanda i systemet \"{a}r ok\"{a}nt. Problemet \"{a}r d\r{a}: Prestanda i systemet \"{a}r ok\"{a}nt. L\"{o}sningen p\r{a} problemet \"{a}r att m\"{a}ta prestanda. Syftet med ditt arbete blir att kartl\"{a}gga systemets prestanda s\r{a} att du f\r{a}r ett m\r{a}tt p\r{a} detta. Fr\r{a}gest\"{a}llningen kan formuleras som: Vad \"{a}r systemets prestanda? Motivationen f\"{o}r arbetet \"{a}r att det \"{a}r viktigt att k\"{a}nna prestanda n\"{a}r systemet ska anv\"{a}ndas f\"{o}r realtidstill\"{a}mpningar. N\"{a}r du har syfte och fr\r{a}gest\"{a}llning klar formulerar du de m\r{a}l som du ska uppfylla med arbetet, i det h\"{a}r fallet kan m\r{a}len t ex vara att m\"{a}ta ett antal olika aspekter av prestanda. Tillsammans kommer dessa m\r{a}l d\r{a} att uppfylla syftet. 
Men ditt examensarbete beh\"{o}ver inte vara formulerat som ett specifikt problem som ska l\"{o}sas. Andra exempel p\r{a} arbeten som kan f\"{o}rekomma som examensarbeten kan vara:
\begin{itemize}
\item[--]	``Case study'' eller studie av n\r{a}got fenomen
\item[--]	Litteraturstudie
\item[--]	Unders\"{o}ka n\r{a}got, t ex hur anv\"{a}ndare interagerar med en mjukvara eller hur en design kan anpassas till en viss grupp anv\"{a}ndare
\item[--]	Analysera t ex j\"{a}mf\"{o}ra prestanda hos olika programvaror
\item[--]	Utv\"{a}rdera h\"{a}nger ofta ihop med att analysera n\r{a}got, din uppgift kan vara att l\"{a}mna en rekommendation om vilket verktyg som b\"{a}st l\"{a}mpar sig f\"{o}r en viss uppgift
\item[--]	Utforska ny teknik eller nya angreppss\"{a}tt. I detta kan ing\r{a} att utveckla en artefakt, t ex en mjukvara eller ett system. 
\item[--]	Utreda en fr\r{a}gest\"{a}llning, t ex genom att g\"{o}ra en f\"{o}rstudie
\item[--]	Utveckla och utv\"{a}rdera en algoritm, t ex f\"{o}r ett ber\"{a}kningsproblem
\end{itemize}
Naturligtvis kan ditt examensarbete ocks\r{a} inneh\r{a}lla flera av ovanst\r{a}ende komponenter. Gemensamt f\"{o}r alla examensarbeten \"{a}r att de ska vara grundligt vetenskapligt f\"{o}rankrade, ett examensarbete f\r{a}r t ex inte vara enbart en implementation.

I m\r{a}nga av exemplen ovan finns det inte n\r{a}got tydligt specificerat problem som ska l\"{o}sas. Det kan ist\"{a}llet r\"{o}ra sig om en fr\r{a}ga som du s\"{o}ker svar p\r{a}, som i exemplet med utv\"{a}rdering. Men alla examensarbeten ska ha syfte, fr\r{a}gest\"{a}llning och motivation. Fr\r{a}gest\"{a}llningen ska vara utformad s\r{a} att den g\r{a}r att besvara p\r{a} n\r{a}got s\"{a}tt genom det arbete du g\"{o}r. Men svaret kan vara abstrakt, det kan t ex vara att bidra till kunskap om fr\r{a}gest\"{a}llningen. I exemplet d\"{a}r uppgiften \"{a}r att utforska en teknik, s\r{a} skulle fr\r{a}gest\"{a}llningen kunna vara ”Vilka problem finns med att utveckla XX”? Det \"{a}r ocks\r{a} vanligt att arbetet inneb\"{a}r att du ska analysera och/eller utv\"{a}rdera artefakten du utvecklat. Fr\r{a}gest\"{a}llning och syfte ska matcha varandra p\r{a} s\r{a} s\"{a}tt att n\"{a}r syftet \"{a}r uppn\r{a}tt s\r{a} besvaras fr\r{a}gest\"{a}llningen.
I texten kommer vi att anv\"{a}nda begreppet uppgift f\"{o}r det du ska g\"{o}ra, vare sig det \"{a}r ett problem som ska l\"{o}sas eller n\r{a}got annat.
Texten i mallen \"{a}r i denna version p\r{a} svenska, men f\"{o}r varje rubrik finns motsvarande engelska ord inom parentes.

Denna sida ska inte ing\r{a} i slutrapporten
\newpage

%add as many preface sections as need by following the "Introduktion" example above.


% ============================= Abstract ==============================
\begin{abstract}
Det h\"{a}r avsnittet ska helt enkelt vara just detta: en sammanfattning av hela rapporten. En l\"{a}mplig omfattning \"{a}r c:a 200 – 250 ord.  En bra tumregel \"{a}r att sammanfattningen ska h\r{a}llas s\r{a} kort det g\r{a}r, den ska vara kompakt men fortfarande tydlig, informativ och v\"{a}cka intresse. Ge de viktigaste fakta
och summera allt det som \"{a}r v\"{a}sentligt i rapporten.  F\"{o}ljande b\"{o}r ing\r{a}:
\begin{itemize}
\item[--]	Presentation/introduktion av omr\r{a}det f\"{o}r arbetet
\item[--]	\"{O}versiktlig presentation av uppgiften inklusive syfte och fr\r{a}gest\"{a}llning
\item[--]	Motivation till varf\"{o}r omr\r{a}det och uppgiften \"{a}r viktiga och intressanta
\item[--]	Generell beskrivning av hur du angripit uppgiften, vad du har gjort
\item[--]	Sammanfattning av resultat och slutsatser och vad ditt arbete bidrar med
\end{itemize}

Inga detaljer ska vara med i sammanfattningen, inte heller beskrivning av hur rapporten \"{a}r uppst\"{a}lld. 
Sammanfattningen ska kunna l\"{a}sas helt frist\r{a}ende fr\r{a}n resten av rapporten, och av en ganska bred grupp av l\"{a}sare. Den ska ge en bra grund f\"{o}r att en l\"{a}sare ska kunna bed\"{o}ma om hen \"{a}r intresserad av att l\"{a}sa hela rapporten. 
Sammanfattningen \"{a}r den del av en rapport som l\"{a}ses allra mest och av flest personer. D\"{a}rf\"{o}r \"{a}r det extra viktigt att du skriver en bra sammanfattning. Du beh\"{o}ver ha ett ordentligt grepp om inneh\r{a}llet i rapporten n\"{a}r du skriver sammanfattningen, och n\"{a}r hela rapporten \"{a}r klar b\"{o}r du granska och vid behov revidera sammanfattningen s\r{a} att den \"{o}verensst\"{a}mmer med rapporten.

\end{abstract}
\newpage


%========================== Table of contents ===========================
\hypersetup{linkcolor=black}
\tableofcontents
%\hypersetup{linkcolor=red}
\vfil
Inneh\r{a}llsf\"{o}rteckningen ska ange de olika rubrikerna i rapporten och p\r{a} vilka sidor i rapporten dessa finns. En f\"{o}rteckning \"{o}ver figurer kan finnas efter inneh\r{a}llsf\"{o}rteckningen. Samtliga figurer/bilder ska vara numrerade och refererade till i texten.

\clearpage

\mainmatter

% ============================== Content: Modify the structure according to your needs ===============================
\section{Inledning (Introduction)}
\label{sec:intro}

This is an example to illustrate the use of the Latex template to write the thesis report. The text for the sections and subsections should be adated to best reflect the content of each section. For example, the "problem formulation" does not need to be a subsection of the introduction. It can be a complete
section on its own (if there is enough material). 

Inledning kan ses som en expanderad version av sammanfattningen. Du kan ha ungef\"{a}r samma struktur men med ett eller tv\r{a} stycken f\"{o}r varje punkt i sammanfattningen. F\"{o}ljande b\"{o}r ing\r{a}:
\begin{itemize}
\item[--]	Presentation av omr\r{a}det och \"{a}mnet f\"{o}r arbetet. Detta b\"{o}r komma tidigt och ska f\r{a}nga intresset. H\"{a}r kan ing\r{a} kort om bakgrund och eventuellt viktiga definitioner av begrepp
\item[--]	Du kan kort beskriva t\"{a}nkt m\r{a}lgrupp f\"{o}r rapporten, vilka har du skrivit f\"{o}r?
\item[--]	Kort \"{o}versikt \"{o}ver tidigare arbeten och dessas begr\"{a}nsningar 
\item[--]	Presentation av uppgiften inklusive syfte och fr\r{a}gest\"{a}llning
\item[--]	Beskrivning av hur du angripit uppgiften, metod och varf\"{o}r denna \"{a}r l\"{a}mplig
\item[--]	Motivation: varf\"{o}r uppgiften \"{a}r intressant, vilka de relevanta fr\r{a}gest\"{a}llningarna \"{a}r, varf\"{o}r ditt angreppss\"{a}tt \"{a}r bra och varf\"{o}r resultaten \"{a}r viktiga.
\item[--]	Beskrivning av de viktigaste resultaten och dessas begr\"{a}nsningar samt vad som \"{a}r nytt i ditt arbete
\item[--]	\"{O}versikt av rapporten
\end{itemize}

Du kan diskutera betydelsen av slutsatserna men inledningen ska bara inneh\r{a}lla kort summering av resultaten. Ingen specialiserad terminologi eller matematik b\"{o}r vara med h\"{a}r. 

Inledningen kan skrivas som en tratt: omr\r{a}de – delomr\r{a}de – uppgift – eventuell deluppgift– syfte. Du leder d\r{a} l\"{a}saren mot en gradvis mer detaljerad och specifik f\"{o}rst\r{a}else av uppgiften och syftet. I slutet av inledningen ska l\"{a}saren och du ha en bas av gemensam f\"{o}rst\r{a}else. L\"{a}saren ska f\"{o}rst\r{a} uppgiften, arbetets omfattning, metod och dess viktigaste bidrag, dvs vad som \"{a}r nytt i ditt arbete. 

\"{A}ven de andra avsnitten i rapporten kan beh\"{o}va en kort inledning i b\"{o}rjan, f\"{o}r att l\"{a}saren ska f\"{o}rst\r{a} syftet med varje avsnitt och dess plats i rapporten.


\subsection{Problemformulering (Problem Formulation)} 

I detta avsnitt formulerar och preciserar du de tre viktiga sakerna syfte, fr\r{a}gest\"{a}llning och motivation. Du ska h\"{a}r presentera uppgiften p\r{a} ett tydligt s\"{a}tt, b\r{a}de p\r{a} h\"{o}g niv\r{a} och i detalj, samt diskutera varf\"{o}r den \"{a}r viktig. Redog\"{o}r f\"{o}r antaganden och begr\"{a}nsningar. Fr\r{a}n beskrivningen av uppgiften kan du sedan formulera syftet och fr\r{a}gest\"{a}llningen. T\"{a}nk p\r{a} att n\"{a}r syftet \"{a}r uppfyllt s\r{a} ska fr\r{a}gest\"{a}llningen kunna besvaras. Det \"{a}r ocks\r{a} viktigt att syfte och motivation h\"{a}nger ihop. N\"{a}r syftet och fr\r{a}gest\"{a}llningen \"{a}r klara kan du b\"{o}rja utveckla m\r{a}len, m\r{a}len ska uppn\r{a}s f\"{o}r att n\r{a} syftet. Varje m\r{a}l ska vara litet, genomf\"{o}rbart och m\"{o}jligt att utv\"{a}rdera.  


\emph{Tips!} Skriv ner din forskningsfr\r{a}ga p\r{a} en lapp som du s\"{a}tter vid sk\"{a}rmen. P\r{a} s\r{a} s\"{a}tt f\r{a}r du hj\"{a}lp att alltid h\r{a}lla forskningsfr\r{a}gan i \r{a}tanke n\"{a}r du arbetar med rapporten.

This is how to use the references~\cite{Berndtsson607210, Blomkvist2014} or~\cite{Turing1950}.




\newpage
\section{Bakgrund (Background)}
\label{sec:background}

I detta avsnitt redog\"{o}r du f\"{o}r s\r{a}dan kunskap som l\"{a}saren beh\"{o}ver f\"{o}r att f\"{o}rst\r{a} ditt arbete och ditt bidrag. Presentera fundamental kunskap som beh\"{o}vs f\"{o}r att f\"{o}rst\r{a} omr\r{a}det och uppgiften. Till exempel kan du h\"{a}r redog\"{o}ra f\"{o}r relevanta teorier och f\"{o}rklara begrepp du anv\"{a}nder eller introducera matematisk notation. Skriv bakgrunden s\r{a} att den som \"{a}r v\"{a}l insatt i omr\r{a}det kan hoppa \"{o}ver den. 

 %SOTA
\newpage
\section{Tidigare arbeten/litteraturöversikt (Related Work)}

Syftet med detta avsnitt \"{a}r att placera in ditt arbete i ett sammanhang och j\"{a}mf\"{o}ra det med tidigare publicerade arbeten och resultat inom omr\r{a}det. Denna del ska vara grundlig. Du beskriver h\"{a}r existerande kunskap och hur denna ut\"{o}kas av ditt arbete. Den ska inneh\r{a}lla analyser av tidigare arbeten som exempelvis beskriver hur olika metoder skiljer sig \r{a}t. Du ska visa p\r{a} de viktigaste likheterna och skillnaderna betr\"{a}ffande uppgift, angreppss\"{a}tt/metodologi samt resultat. Det \"{a}r viktigt att du p\r{a} ett neutralt s\"{a}tt diskuterar f\"{o}r- och nackdelar med ditt eget arbete j\"{a}mf\"{o}rt med andras.

Detta skapar ocks\r{a} en f\"{o}rv\"{a}ntan p\r{a} bidraget f\"{o}r ditt arbete, l\"{a}saren l\"{a}r sig h\"{a}r om begr\"{a}nsningar hos tidigare arbeten och varf\"{o}r din uppgift \"{a}r en utmaning.. 

Tillsammans kommer detta avsnitt tillsammans med bakgrund att introducera ”state of the art”/”state of practice” och dess brister, betydelsen av uppgiften samt vad ditt arbete ska j\"{a}mf\"{o}ras med.

\newpage
\section{Metod (Method)} 
\label{sec:method}

I det h\"{a}r avsnittet ska du beskriva vilka vetenskapliga metoder du har anv\"{a}nt och hur du har g\r{a}tt tillv\"{a}ga med sj\"{a}lva arbetet. F\"{o}r varje m\r{a}l ovan identifierar du en metod f\"{o}r att n\r{a} m\r{a}let. Valen av metod ska motiveras. Du kan t ex ha gjort en matematisk modell, anv\"{a}nt simuleringar, gjort en implementation som du testat eller gjort experiment som du kanske utv\"{a}rderat med hj\"{a}lp av statistiska metoder. Vi avser h\"{a}r i f\"{o}rsta hand att du beskriver de vetenskapliga metoder du anv\"{a}nt, men det \"{a}r ocks\r{a} bra om du ger en beskrivning av hur du arbetat med uppgiften. Avsnittet Metod svarar ocks\r{a} p\r{a} varf\"{o}r du gjorde p\r{a} ett visst s\"{a}tt eller varf\"{o}r du anv\"{a}nde ett visst verktyg. Du ska allts\r{a} inte bara beskriva ”vad” utan  ocks\r{a} ”varf\"{o}r”. St\"{a}ll dig fr\r{a}gan: kan den valda metoden hj\"{a}lpa mig att n\r{a} de uppsatta m\r{a}len och d\"{a}rmed besvara fr\r{a}gest\"{a}llningen?

Att v\"{a}lja r\"{a}tt vetenskapliga metod(er) \"{a}r viktigt f\"{o}r att du ska n\r{a} dina m\r{a}l, d\"{a}rf\"{o}r \"{a}r detta en punkt som du p\r{a} ett tidigt stadium b\"{o}r diskutera med din handledare. S\"{o}k ocks\r{a} i litteraturen efter bra beskrivningar av metoder, och hur du p\r{a} b\"{a}sta s\"{a}tt skriver ett Metodavsnitt. 

\newpage
\section{Etik och Samhälleliga aspekter (Ethical and Societal Considerations)}

I f\"{o}rst hand avser vi med ”etik” h\"{a}r forskningsetiska fr\r{a}gor. Inneb\"{a}r ditt val av fr\r{a}gest\"{a}llning eller metod n\r{a}got forskningsetiskt st\"{a}llningstagande? Om du till exempel intervjuar personer f\"{o}r ditt arbete, kan du garantera dessa anonymitet och p\r{a} vilket s\"{a}tt anv\"{a}nder du den information du f\r{a}r av dem? Finns det andra etiska aspekter att beakta i arbetet? Kan det finnas etiska aspekter p\r{a} resultatet av ditt arbete? Du b\"{o}r tydligt ange om du anser att ditt arbete inte inneh\r{a}ller n\r{a}gra forskningsetiska fr\r{a}gor. 

Du ska ocks\r{a} kritiskt granska och analysera ditt arbete med h\"{a}nsyn till samh\"{a}lleliga aspekter. H\"{a}r kan du till exempel diskutera hur ditt arbete f\"{o}rh\r{a}ller sig till m\r{a}l som ekonomisk, social och ekologiskt h\r{a}llbar utveckling. Det kan ocks\r{a} finnas juridiska och politiska aspekter p\r{a} ditt arbete. 

\newpage
\section{``Beskrivning av arbetet''/ ``Description of the work'') }

Efter avsnitten ovan f\"{o}ljer nu en beskrivning av vad du gjort. Du ska inte anv\"{a}nda rubriken ovan, utan ers\"{a}tta den med l\"{a}mpliga rubriker, beroende p\r{a} ditt arbete. Strukturen ska g\"{o}ras tydlig genom avsnittsrubrikerna. Det \"{a}r viktigt med en klar och tydlig logisk struktur och ett ber\"{a}ttande fl\"{o}de. Du ska ha med avancerad bakgrundskunskap som \"{a}r n\"{o}dv\"{a}ndig f\"{o}r att f\"{o}rst\r{a} hur du l\"{o}st uppgiften, och definiera hypoteser och viktiga begrepp. Beskrivning av experiment ska vara s\r{a}dan att det ska g\r{a} att upprepa experimenten. Om en s\r{a}dan beskrivning blir v\"{a}ldigt l\r{a}ng och detaljerad kan du l\"{a}gga den i en bilaga, se nedan.  



\newpage
\section{Resultat (Results)}

H\"{a}r kan du till exempel presentera resultat av experiment, bevis, analys av data etc. Dina resultat m\r{a}ste beskrivas s\r{a} tydligt att en l\"{a}sare kan bed\"{o}ma dem.  Du ska ocks\r{a} f\"{o}rklara och analysera resultaten.



\newpage
\section{Diskussion (Discussion)}

H\"{a}r presenterar du tolkning av resultaten och bed\"{o}mer deras signifikans. Diskutera m\"{o}jliga konsekvenser av resultaten, och presentera eventuella rekommendationer. Det \"{a}r viktigt att du redog\"{o}r f\"{o}r om du uppn\r{a}tt de m\r{a}l du satte upp och d\"{a}rmed besvarat din fr\r{a}gest\"{a}llning och uppn\r{a}tt syftet med arbetet. Avsnittet ska ocks\r{a} inneh\r{a}lla reflektioner kring arbetet, som till exempel dess begr\"{a}nsningar.  Du kan ocks\r{a} diskutera l\"{o}sningar p\r{a} problem som du identifierat och diskuterat tidigare, eller ta upp andra problem som arbetet inte behandlat, fr\r{a}gor som ej besvarats. Koppla ocks\r{a} dina resultat till tidigare arbeten. P\r{a} s\r{a} s\"{a}tt kan diskussionen bli ett samtal med det du skrev i tidigare avsnitt.  Slutligen ska du s\"{a}tta ditt eget arbete i ett st\"{o}rre sammanhang, bredda ditt perspektiv. Kan dina resultatet generaliseras? Kan det du gjort anv\"{a}yndas i n\r{a}ygot annat sammanhang? 

\newpage
\input{./conclusions}
\newpage

%\input{./Acknowledgments}


% ============================= References ============================
%\newpage
\bibliographystyle{IEEEtran}
\bibliography{./references}
\addcontentsline{toc}{section}{References}

% ============================ Appendices =============================
%\newpage
%\begin{appendices}

	%\input{./Appendices/appendices1}
	%\clearpage
	
	%\input{./Appendices/appendices2}
	%\clearpage
	
%\end{appendices}

\end{document}
